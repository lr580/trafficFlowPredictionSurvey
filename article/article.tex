\documentclass[conference]{IEEEtran}

% correct bad hyphenation here
\hyphenation{op-tical net-works semi-conduc-tor}

\begin{document}

\title{A Survey on Modern Deep Learning\\ for Traffic Flow Prediction}

\author{\IEEEauthorblockN{Zixin Qin}
\IEEEauthorblockA{School of Artificial Intelligence\\
Sun Yat-sen University\\
% Atlanta, Georgia 30332--0250\\
Email: qinzx5@mail2.sysu.edu.cn}
% \and
% \IEEEauthorblockN{Homer Simpson}
% \IEEEauthorblockA{Twentieth Century Fox\\
% Springfield, USA\\
% Email: homer@thesimpsons.com}
}

% make the title area
\maketitle

\begin{abstract}
The abstract goes here.
\end{abstract}

% no keywords

% For peer review papers, you can put extra information on the cover
% page as needed:
% \ifCLASSOPTIONpeerreview
% \begin{center} \bfseries EDICS Category: 3-BBND \end{center}
% \fi
%
% For peerreview papers, this IEEEtran command inserts a page break and
% creates the second title. It will be ignored for other modes.
\IEEEpeerreviewmaketitle

\section{Introduction}

This demo file is intended to serve as a ``starter file''
for IEEE conference papers produced under \LaTeX\ using
IEEEtran.cls version 1.8b and later.\cite{jiang_graph_2022}

I wish you the best of success.\cite{vlahogianni_short-term_2014}

\subsection{Subsection Heading Here}
Subsection text here.\cite{zhu_big_2019}

\subsubsection{Subsubsection Heading Here}
Subsubsection text here.

\section{Background}

\section{Deep Learning for Traffic Flow Prediction}

\section{Datasets for Traffic Flow Prediction}

\section{Future Directions}

\section{Experiments}

\section{Conclusion}
The conclusion goes here.

\bibliographystyle{IEEEtran}
\bibliography{thesis}

\section*{Acknowledgment}

The authors would like to thank...

\end{document}


