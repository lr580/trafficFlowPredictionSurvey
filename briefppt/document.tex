%%%%%%%%%%%%%%%%%%%%%%%%%%%%%%%%%%%%%%%%%%%%%%%%%%%%%%%%%%%%%%%%%%%%%%%%%%%%%%%%%%%
%% This project aims to create the UNAL template for presentation.                %%
%% author:Félix Julián Gutiérrez 
%% contacts:                                                                     %%
%%    e-mail: fjgutierrezb@unal.edu.co                                                   %%
%%   www.unal.edu.co
%%
%%%%%%%%%%%%%%%%%%%%%%%%%%%%%%%%%%%%%%%%%%%%%%%%%%%%%%%%%%%%%%%%%%%%%%%%%%%%%%%%%%%
\documentclass{libs/format}
% Inserting the preamble file with the packages
\input{libs/preamble.tex}
% Inserting the references file
\bibliography{references.bib}


% 演示文稿标题:
%  1 底部显示的标题;
%  2 标题页正中显示的大标题;
\title[sysu ppt template]{中山大学PPT模板a}

% 可选:标题页显示的副标题
\subtitle{使用beamer写PPT}

% 学校学院信息:
%  1 顶部学校信息
%  2 标题页显示的学院学校信息
\institute[Sun Yat-Sen University]{
  计算机学院 \\ 
  中山大学}

%% 时间信息
%  1 底部显示的标准时间信息
%  2 标题页显示的中文时间信息
\date[\ctoday]{\ctoday}

\begin{document}

\input{libs/code_style.tex}

% 作者信息
\author[Nelson Cheung]{
  Nelson Cheung \\\medskip
  {\email{zhangjunyu@nelson-cheung.cn}} \\
  {\url{nelson-cheung.cn}} \\
  {\small{License: \href{https://license.coscl.org.cn/MulanPSL2/}{MulanPSL-2.0}}}
}

\begin{frame}
  % 标题页
  \titlepage
\end{frame}
% 标题页不编号
\setcounter{framenumber}{0}

\begin{frame}
  \tableofcontents[sectionstyle=show,subsectionstyle=show/shaded/hide,subsubsectionstyle=show/shaded/hide]
  \addtocounter{framenumber}{-1}
\end{frame}

\AtBeginSection[]
{
	\begin{frame}{主要内容}
		\tableofcontents[sectionstyle=show/shaded,subsectionstyle=show/shaded/hide]
		\addtocounter{framenumber}{-1}
	\end{frame}
}


\section{引言}

\begin{frame}{写在前面}
  本模板基于Presentación UNAL\footnote[frame]{https://www.overleaf.com/latex/templates/presentacion-unal/nzfkbkgnctfp},SYSU Beamer Template\footnote[frame]{https://github.com/yxnchen/sysu-beamer-template},THU Beamer Theme\footnote[frame]{https://www.overleaf.com/latex/templates/thu-beamer-theme/vwnqmzndvwyb},做出如下修改。
  \begin{itemize}
    \item 去除背景的水印图片,背景水印图片会导致页面切换时产生屏闪现象。
    \item 基于Presentación UNAL的排版风格,使用SYSU Beamer Template的颜色主题,采用THU Beamer Theme和Presentación UNAL的内容进行测试。
    \item 本模板使用MulanPSL-2.0开源许可协议。
  \end{itemize}
\end{frame}

\begin{frame}[fragile]{编译方法}
  本模板在texlive 2021下成功编译, 依次执行下面4条命令即可编译。
  \begin{lstlisting}
xelatex document.tex
bibtex document
xelatex document.tex
xelatex document.tex\end{lstlisting}
  清除编译生成的中间文件。
  \begin{lstlisting}
make clean\end{lstlisting}  

\end{frame}
\section{使用举例}
\begin{frame}{blocks}
  \begin{block}{国内\LaTeX\ 讨论区}
    \begin{enumerate}
      \item LaTeX Studio\footnote{\url{https://www.latexstudio.net/}}
    \end{enumerate}
  \end{block}

  \begin{alertblock}{国外\LaTeX\ 讨论区}
    \begin{enumerate}
      \item LaTeX Stack Exchange\footnote{\url{https://tex.stackexchange.com/}}
    \end{enumerate}
  \end{alertblock}

  \begin{exampleblock}{example 示例环境}
    \begin{enumerate}
      \item LaTeX Stack Exchange\footnote{\url{https://tex.stackexchange.com/}}
    \end{enumerate}   
  \end{exampleblock}

\end{frame}

\begin{frame}{表格}
  \begin{itemize}
      \item \LaTeX 广泛用于学术界,期刊会议论文模板
  \end{itemize}
  \begin{table}[h]
      \centering
      \begin{tabular}{c|c}
          Microsoft\textsuperscript{\textregistered}  Word & \LaTeX \\
          \hline
          文字处理工具 & 专业排版软件 \\
          容易上手,简单直观 & 容易上手 \\
          所见即所得 & 所见即所想,所想即所得 \\
          高级功能不易掌握 & 进阶难,但一般用不到 \\
          处理长文档需要丰富经验 & 和短文档处理基本无异 \\
          花费大量时间调格式 & 无需担心格式,专心作者内容 \\
          公式排版差强人意 & 尤其擅长公式排版 \\
          二进制格式,兼容性差 & 文本文件,易读、稳定 \\
          付费商业许可 & 自由免费使用 \\
      \end{tabular}
  \end{table}
\end{frame}

\begin{frame}{图片}
  \begin{figure}
      \centering
      \caption{Logo of SYSU}
      \includegraphics[scale=0.05]{libs/sysu_big_logo.jpeg}
      \source{Nelson Cheung\cite{sitiounal} \cite{einstein}}
  \end{figure}
\end{frame}

\begin{frame}{数学公式}
  \begin{equation*}
    J(\theta) = \mathbb{E}_{\pi_\theta}[G_t] = \sum_{s\in\mathcal{S}} d^\pi (s)V^\pi(s)=\sum_{s\in\mathcal{S}} d^\pi(s)\sum_{a\in\mathcal{A}}\pi_\theta(a|s)Q^\pi(s,a)
  \end{equation*}
      
  \begin{align}
    Q_\mathrm{target}&=r+\gamma Q^\pi(s^\prime, \pi_\theta(s^\prime)+\epsilon)\\
          \epsilon&\sim\mathrm{clip}(\mathcal{N}(0, \sigma), -c, c)\nonumber
  \end{align}
\end{frame}

\begin{frame}{数学公式}
  \begin{exampleblock}{编号多行公式}
      % Taken from Mathmode.tex
      \begin{multline}
          A=\lim_{n\rightarrow\infty}\Delta x\left(a^{2}+\left(a^{2}+2a\Delta x+\left(\Delta x\right)^{2}\right)\right.\label{eq:reset}\\
          +\left(a^{2}+2\cdot2a\Delta x+2^{2}\left(\Delta x\right)^{2}\right)\\
          +\left(a^{2}+2\cdot3a\Delta x+3^{2}\left(\Delta x\right)^{2}\right)\\
          +\ldots\\
          \left.+\left(a^{2}+2\cdot(n-1)a\Delta x+(n-1)^{2}\left(\Delta x\right)^{2}\right)\right)\\
          =\frac{1}{3}\left(b^{3}-a^{3}\right)
      \end{multline}
  \end{exampleblock}
\end{frame}

\begin{frame}{伪代码}
  \begin{algorithm}[H]
      \SetAlgoLined
      \LinesNumbered
      \SetKwInOut{Input}{input}
      \SetKwInOut{Output}{output}
      \Input{x: float, y: float}
      \Output{r: float}
      \While{True}{
        r = x + y\;
        \eIf{r >= 30}{
         ``do something''
         break\;
         }{
          ``do nothing''
        }
       } 
       \caption{Algorithm Example}
  \end{algorithm}
\end{frame}

\begin{frame}{真实代码}
  \lstset{language=Python}
  \lstinputlisting[language=Python]{code/main.py}
\end{frame}

\begin{frame}{真实代码}
  \lstinputlisting[language=C]{code/source.c}
\end{frame}

\begin{frame}{真实代码}
  \lstinputlisting[language=Java]{code/helloworld.java}
\end{frame}

\begin{frame}{真实代码}
  \lstinputlisting[language=HTML]{code/index.html}
\end{frame}

\section{存在的问题}

\begin{frame}[fragile]
若在使用过程中发现问题,可以提PR。
\begin{enumerate}
  \item (已解决)frame中脚注不显示的问题,使用\verb|\footnote[frame]{}|来添加脚注。
  \item (已解决)overleaf无法编译。
\end{enumerate}

\end{frame}

\begin{frame}[allowframebreaks]
    \printbibliography[title = {参考文献}]
\end{frame}

\end{document}