\documentclass{libs/format}
\usepackage{hyperref}
\input{libs/preamble.tex}
\setbeamertemplate{caption}[numbered]
\bibliography{../article/thesis.bib}
\AtEveryBibitem{\clearfield{url}}  % 不显示 URL
\AtEveryBibitem{\clearfield{doi}}
\AtEveryBibitem{\clearfield{issn}}

\title[交通流量预测综述]{基于深度学习的交通车流量预测综述}

% 可选:标题页显示的副标题
\subtitle{A Survey on Modern Deep Learning for Traffic Flow Prediction}

\institute[Sun Yat-Sen University]{\normalsize 中山大学人工智能学院}

\date[\ctoday]{\ctoday}

\begin{document}

\input{libs/code_style.tex}

% 作者信息
\author[]{\large 覃梓鑫}

\begin{frame}
  % 标题页
  \titlepage
\end{frame}
% 标题页不编号
% \setcounter{framenumber}{0}

% \begin{frame}
%   \tableofcontents[sectionstyle=show,subsectionstyle=show/shaded/hide,subsubsectionstyle=show/shaded/hide]
%   \addtocounter{framenumber}{-1}
% \end{frame}
\AtBeginSection[]
{
	\begin{frame}{目录}
		\tableofcontents[currentsection, subsectionstyle=shaded]
    \addtocounter{framenumber}{-1}
	\end{frame}
}
\AtBeginSubsection[]
{
	\begin{frame}{章节目录}
		\tableofcontents[currentsection, currentsubsection]
    \addtocounter{framenumber}{-1}
	\end{frame}
}

\section{介绍}

\begin{frame}{问题定义}
  \begin{block}{交通流量预测}
    交通流量预测是使用可学习的函数,利用历史交通流量数据输入,预测未来交通流量。\cite{T-ZS1}

    其中,交通流量可以表示为在一段时间内一个空间节点经过的车辆数目。例如北门在今天 $17:00\sim17:05$ 间经过的车辆为 $58$ 辆,则流量为 $58$。

    设共有 $n$ 个空间节点,且已知 $t$ 个时间段,则输入数据大小为 $X\in R^{n\times t}$,并预测全体 $n$ 个节点在接下来 $t'$ 个时间段的流量。显然,这是一个时空序列预测任务。

    % 即:找到模型参数,最小化误差:
    % \[
    %   \hat y_{t+T'}=f([X_{t-T+1},X_{t-T},\cdots,X_t]),\quad \theta^*=\arg\min_{\theta^*}L(y_{t+T'},\hat y_{t+T'};\theta^*)
    % \]
    % 其中 $X$ 是输入数据,$y_t$ 是时间 $t$ 的观察值,$\hat y_t$ 是预测值,$T$ 是输入序列长度,$T'$ 是预测范围,$L$ 是损失函数,$f$ 是任意函数,$\theta^*$ 是最优参数。
  \end{block}
  交通流量预测对拥挤控制、红绿灯控制、导航等具有重要意义。\cite{T-ZS2, T-ZS32}
  
  按目标类别,流量可分为车流量、人流量、自行车流量等。本文只研究交通机动车流量预测。
\end{frame}

\begin{frame}{空间建模}
  对空间节点常用有两种建模方式:

  \begin{description}
    \item[网格节点] 将一个地理区域表示为 $r$ 行 $c$ 列的网格(通常 $8\le r$, $c\le 32$),每个单位区域的长宽通常为数百米内,且接近正方形。
    \item[图节点] 将地理区域表示成街道或路口等的点集 $V$,点之间的关联表示成边集 $E$ 或矩阵 $A$(如邻接、距离等矩阵),构成一张有向或无向图 $G$。通常是数百节点的稀疏图。
  \end{description}

  两种节点如下图\ref{fig:grid-vs-graph}所示。
  \begin{figure}
    \centering
    \includegraphics[height=80pt]{img/grid-vs-graph.png}
    \caption{网格节点与图节点对比\cite{T-ZS30}}\label{fig:grid-vs-graph}
    % 或 T-ZS31 的图
  \end{figure}
\end{frame}

% \begin{frame}{小例子} % toy example
%   以 PEMSD7 数据集(共 $883$ 个图节点,$28224$ 个时间段)为例,取小部分数据展示如下:
% \end{frame}

\begin{frame}{特点}
  交通流量预测具有如下特点:\cite{T-ZS1, T-ZS2}
  \begin{enumerate}
    \item 存在全局外部因素,如时间、天气、节假日、交通事故等。
    \item 地点间存在相互影响,如一条道路塞车会在一定时间后让它附近的道路也塞车。
    \item 比其他时间序列任务的数据量和维度更大。
  \end{enumerate}
  \begin{exampleblock}{示例}
    例,如图\ref{fig:space-corr}所示,道路1(蓝色)与道路2(红色)是同一条路的两个路段,时间序列相似;道路3(黄色)与道路1方向相反。
  \end{exampleblock}
  \begin{figure}
    \centering
    \includegraphics[height=80pt]{img/space-corr.png}
    \caption{道路的空间关联\cite{T-ZS40}}\label{fig:space-corr}
    % 但这个图是 speed 的
  \end{figure}
\end{frame}

\begin{frame}{主流技术}
  早期主要使用统计模型(如 ARIMA)与机器学习(如 kNN、SVR)。\cite{T-ZS1}

  现在主要使用深度学习模型,可以按时间特征和空间特征处理来讨论。
  \begin{itemize}
    \item 对时间特征,主要使用 RNN,例如 GRU、LSTM、BiLSTM 等,可以修改 LSTM 密度核为卷积、GRU 矩阵乘法为扩散卷积等。
    
    也可以使用 CNN、注意力机制、Transformer 等技术处理。
    \item 对空间特征:针对网格节点,主要使用 CNN,如 1/2/3D CNN 的多个变式;针对图节点,主要使用 GNN,有图卷积、图注意力、图自编码器、图 GAN 等。
  \end{itemize}
\end{frame}

\begin{frame}{相关类似问题}
  下列问题的场景与技术与交通车流量预测相关,简要介绍如下。
  \begin{itemize}
    \item 人流量预测:与车流量预测,但一般是对地铁口等场景,预测人群流入/流出量。\cite{T-ZS44}
    \item 自行车流量预测:与车流量预测类似,数据集来源为共享自行车\cite{T-162}、E-scooter\cite{T-286} 等非机动车。
    \item 交通速度预测:对某时间道路车辆的平均速度进行预测。多数交通流量预测模型可直接用于速度预测,反之亦然。
    \item 到达时间预测:即 ETA(estimated time of arrival) 预测。如给定车辆 ID、出发时间、行驶路线,预测到达时间。\cite{T-316}
    \item 交通需求预测:如某时某区域对打车需求量的预测。
  \end{itemize}
\end{frame}

\section{主流技术}

\subsection{时间方面}

\begin{frame}{写2}
  abc
\end{frame}

\subsection{空间方面}

\begin{frame}{写22}
  abc
\end{frame}

\section{数据集}

\begin{frame}{概述}
  数据集由时间序列数据、空间信息数据、其他辅助数据三部分组成。
  \begin{description}
    \item[时间序列数据] 核心数据。由各节点历史交通流量数据组成。通常各节点的各时间段长度相等(一般为 $5$ 分钟),总时间跨度一般在数个月到一两年不等。常用数值矩阵表示。
    \item[空间信息数据] 网格节点或图节点(见上文空间建模)。常用边集等方式表示。
    \item[其他辅助数据] 主要包括日期时间、天气、交通异常事件等。将这些辅助信息作为补充信息提供给模型,可以
    提升预测效果。
  \end{description}
  本文只介绍交通车流量相关的数据集。
\end{frame}

\begin{frame}{时间序列数据来源}
  主要来源有两种,分别是探测器定点数据和 GPS 轨迹数据。特点如下。
  \begin{description}
    \item[定点数据] 该类数据来源权威质量高、预处理简单、常表示为图结构。其缺点是采集成本高昂、覆盖范围小、难以转化为网格数据。代表例子是 PeMS 数据集。
    \item[轨迹数据] 该类数据覆盖道路广、可视化简单、常转换为网格结构。但预处理复杂、数据质量较差,且可能统计不充分(一条道路上经过的全部车辆只有部分轨迹被记录,则记录流量偏小)。代表例子是 TaxiNYC、DIDI GAIA 数据集。
  \end{description}
\end{frame}

\begin{frame}{经典数据集-PeMS}
  PeMS(Caltrans \textbf{Pe}rformance \textbf{M}easurement \textbf{S}ystem)\cite{T-81}是研究最广泛的数据集之一,由美国加利福尼亚州主要公路的超过 $39000$ 个探测器收集而来,并从 $2001$ 年开始持续不断收集数据至今。这些数据每半分钟收集一次,并将其聚合成 $5$ 分钟一个时间段或其他粒度的数据。研究者主要使用 PeMS 的一些代表性子集,如PeMSD3、PeMSD4、PeMSD7、PeMSD8。如下表\ref{table:pems}所示。

  \setlength{\tabcolsep}{2pt}
  \begin{footnotesize}
  \begin{table}[]  
    \centering\caption{PeMS 代表性子集}\label{table:pems}
    \begin{tabular}{|c|c|c|c|c|c|}  
    \hline  
    数据集名称 & 地点 & 时间跨度 & 时间段数 & 探测器点数 & 边数 \\ \hline  
    PeMSD3 & 中北部区域 & $2018/9/1\sim2018/11/30$ & $26208$ & $358$ & $547$ \\ \hline  
    PeMSD4 & 旧金山湾区 & $2018/1/1\sim2018/2/28$ & $16992$ & $307$ & $340$ \\ \hline  
    PeMSD7 & 洛杉矶区域 & $2017/5/1\sim2017/8/31$ & $28224$ & $883$ & $866$ \\ \hline  
    PeMSD8 & 圣贝纳迪诺区 & $2016/7/1\sim2016/8/31$ & $17856$ & $170$ & $277$ \\ \hline  
    \end{tabular}  
  \end{table}
  \end{footnotesize}
  
\end{frame}

\begin{frame}{经典数据集-TaxiBJ}
  TaxiBJ\cite{T-51}提供了中国北京超过 $34000$ 辆出租车在下面四个时间段每 $30$ 分钟一次共 $22459$ 个有效时间段的 GPS 轨迹数据:(1)$2013/7/1\sim2013/10/30$、(2)$2014/3/1\sim2014/6/30$、(3)$2015/3/1\sim2015/6/30$、(4)$2015/11/1\sim2016/4/10$。
  
  将数据建模为了 $32\times32=1024$ 的网格节点,提供交通流量流入和流出数据。还提供了节假日信息、16 种天气状况、温度、风速的额外信息。许多研究者会使用该数据集的部分或全部来研究北京的出租车。

  选取其中编号为 $100$ 的点(即下标第 $3$ 行第 $4$ 列的网格节点),对该节点验证集的前 $5$ 个时间点的流量展示如下表\ref{table:taxibj}所示。

  \begin{table}[]  
    \centering\caption{TaxiBJ 数据样本部分展示}\label{table:taxibj}
    \begin{tabular}{|c|c|c|c|c|c|}  
    \hline  
    时间编号 & 0 & 1 & 2 & 3 & 4 \\ \hline  
    流入流量 & 57.0 & 54.0 & 51.0 & 58.0 & 59.0 \\ \hline  
    流出流量 & 55.0 & 53.0 & 50.0 & 59.0 & 58.0 \\ \hline  
    \end{tabular}  
  \end{table}
\end{frame}

\begin{frame}{其他时间序列数据集}
  
\end{frame}

\begin{frame}{其他组成成分}
  
\end{frame}

\section{未来方向}

\begin{frame}{写4}
  abc
\end{frame}

\section{结论}

\begin{frame}{写5}
  abc
\end{frame}

\begin{frame}[allowframebreaks, noframenumbering]
    \printbibliography[title = {参考文献}]
\end{frame}

\end{document}