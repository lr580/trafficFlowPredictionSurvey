% 由于需求变化,目前停止编写。

\documentclass{ctexart}
\usepackage{hyperref}
\usepackage[numbers, super]{natbib}
\usepackage{longtable}
\usepackage{tabu}
\usepackage{makecell}
% \usepackage{graphicx}
% \usepackage{tabularx}
\renewcommand{\cite}[1]{\textsuperscript{[\citenum{#1}]}}
\begin{document}
\title{交通流量预测数据集综述}
\author{中山大学}
\date{\today}
\maketitle
\tableofcontents
\section{概述}

交通流量预测是一种时空序列预测任务,其数据集主要由时间序列数据、空间信息数据和其他辅助数据三部分组成。

\begin{description}
    \item[空间信息数据] 主要为真实世界的道路结构进行建模提供必要信息。根据建模方式不同,可以将其分为网格空间信息和拓扑空间信息。无论是哪一种信息,都以节点作为基本单位。其中,网格空间信息将道路结构划分为若干个网格节点,每个节点表示一小个矩形区域,多个节点排列成二维矩阵来表示现实世界的一片区域。而拓扑空间信息则将道路视作有向图或无向图,图中的每个图节点可以表示一条道路、一个探测器、一个路口等。下图 展示了两种建模的空间信息示意图。%TODO 提供图片
    \item[时间序列数据] 是交通流量预测的核心数据,由各节点的历史交通流量数据组成,每个节点通常按等间隔地记录过去的一段时间每单位时间段经过该节点的车流量大小数值。一般而言,每个节点的时间跨度、间隔完全一致,故时间段数相同。因此,通常将其表示为一个数值矩阵,行数为节点数量,列数为时间段的数目。在预测任务中,通常利用所有节点的多段过去的流量数据,来预测全体节点未来一段或多段的交通流量。
    \item[其他辅助数据] 这主要包括日期时间、天气、交通异常事件等辅助信息。这是由于节假日与工作日(不同日期)、早晚高峰与非高峰时段(不同时间)、晴天与暴雨天(不同天气)、交通事故等异常事件都会对交通流量产生影响。因此,将这些辅助信息作为补充信息提供给模型,可以提升预测效果。
\end{description}

这些数据集组成成分会存在相互影响。例如,时间序列数据与空间信息数据存在密不可分的关联,例如,两条道路相连通时,一条道路的堵塞,会在一段时间内造成另一条道路也发生堵塞。因此,当预测某个节点的未来交通流量时,不能只考虑该节点的时间序列数据,还要联系空间信息找到相关联的其他节点,将这些关联节点的时间序列数据也共同纳入考虑。

在本文的剩余部分,将详细介绍这三种组成成分,分析它们的特征。随后对每种成分,具体介绍国内外的常用数据集,陈述这些数据集的特点,并列表格详细展示本文所搜集的数据集及其相关统计。

\section{空间信息数据介绍}
空间信息数据有多种划分方式,例如,Tedjopurnomo 等人\cite{T-ZS1} 提出,按照数据收集的方式,可以将数据分为点数据集和轨迹数据集。其中,点数据集是由探测器等固定不动的仪器从固定点收集的数据,而轨迹数据为使用 GPS 等定位技术在车辆上收集的数据。而 Jiang 和 Luo\cite{T-ZS2} 提出,按照道路的类型和粒度可以把交通流量数据分为道路级数据、区域级数据、站点级数据。而在本文中,将交通流量数据里的空间信息数据按照空间结构划分为了欧氏空间的网格型数据和非欧空间的拓扑型数据。

原始的交通路网数据可以直接从在线地图服务、政府部门网站等获得,如 OpenStreetMap\footnote{\url{https://www.openstreetmap.org}}。通常会将原始数据进行预处理,然后与时间序列数据等其他组成成分整理并形成公开数据集,并发布在网络上,例如 GitHub\footnote{\url{https://github.com/}} 和 PaperWithCode\footnote{\url{https://paperswithcode.com/task/traffic-prediction}}。

% \subsection{网格空间信息数据}
对网格空间,可以表示为一个矩阵,每个元素代表一片区域,每个区域大小相等,通常是正方形区域(TODO)。形式化地说,可以表示成 $r$ 行 $c$ 列的网格 $G$,且一般 $r,c$ 较为接近,即会选取一片正方形区域做为网格空间。通常网格数据使用 CNN 来做流量预测应用。

% \subsection{拓扑空间信息数据}
对拓扑空间,一张图 $G$ 由点集 $V$ 和边集 $E$ 组成,边集可以存储成邻接表等形式,更多地通过邻接矩阵 $A$ 的方式存储,因为矩阵更适合模型运算。根据边是否为单向的,可以分为有向图和无向图。边通常具有权重信息,如布尔值 $1$ 代表连通、$0$ 代表不连通,或者使用连续数值表示两点间的距离、关联程度或其他信息。常常使用 GNN 来实现拓扑空间数据的流量预测。%多数数据集将图视为静态的,也有些数据集是随时间而动态变化的。?
%注意cite来引用这些例子

\section{时间序列数据集介绍}
时间序列数据集的来源非常丰富,涵盖多个国内外主要城市,且许多地点有不止一个数据集。并且,对多数常用数据集,即使是同一个数据集,也有多个子集。这些子集通常有不同的时间范围与地点范围,也有可能收集不同的交通信息。它们通常是由原始数据集预处理而来,形成公开数据集并被多数研究者使用。

时间序列数据集可以按照流量主体进一步分类,分为车流量、人流量、自行车流量。这里只列举车流量数据集,其他交通流量的数据集可以参考这些论文\cite{T-ZS1, T-ZS2, T-ZS29}。而车流量数据根据来源也可以细分为道路探测器数据、出租车数据、(TODO)等。
% more cite?

形式化来说,设有 $n$ 个节点,每个节点均有 $t$ 个时间段的流量数据,则时间序列数据集可以表示为一个 $n \times t$ 的矩阵。在多数数据集里,每个时间段的长度为 $5$ 分钟,也有一些是 $1$ 小时或其他长度的。各时间段的总时长多数在数个月到一两年内。大部分数据集涵盖了全天候数据,也有些数据集缺乏深夜时段、节假日数据等。数据集的节点数量一般在数百到数千不等。具体情况可以参见下文表格统计。

% 流量还可以细分为流入流量和流出流量(inflow and outflow),当空间被建模为网格时,常将流量分别表示为流入和流出流量,即对于每个网格节点的每个时间段,分别使用 $2$ 个数值记录流入流量与流出流量。但在拓扑图节点数据集上,通常不会区分流入与流出流量\cite{T-ZS45}。

下面首先将详细介绍常用的数据集,列表统计本文收集的全部相关数据集,最后对数据集成分和特征及其对预测任务的影响作简单的讨论。

\subsection{常用数据集介绍}
\subsubsection{PeMS}
PeMS(Caltrans Performance Measurement System)\footnote{\url{http://pems.dot.ca.gov/}\label{pems}}是研究最广泛的数据集之一,由美国加利福尼亚州主要公路的超过 $39000$ 个探测器收集而来,并从 $2001$ 年开始持续不断收集数据至今。这些数据每半分钟收集一次\cite{T-ZS2},并将其聚合成 $5$ 分钟一个时间段或其他粒度的数据\cite{T-ZS1}。常用的站点 $5$ 分钟数据包括时间戳(时间段的开始时刻,格式为 $24$ 小时制的 \verb|MM/DD/YYYY HH:MI:SS|)、总流量、流量加权平均速度、平均占用率、车道类型、行驶方向、高速公路号等多列组成\footnote{\url{https://pems.dot.ca.gov/?dnode=Clearinghouse&type=station_5min&district_id=3&submit=Submit}}。截至 $2024$ 年 $10$ 月为止,共有 $19601$ 个站点、$41236$ 里的路程\textsuperscript{\ref{pems}}。

PeMS 数据集包含的站点和时间都比较庞大,所以研究者主要使用 PeMS 的一些代表性子集。可以看到,PeMS 数据列中既包含流量又包含速度,所以有一些子集,如 PeMS-BAY\footnote{\url{https://github.com/liyaguang/DCRNN}\label{pems-bay}}, PeMSD7M\footnote{\url{https://github.com/VeritasYin/STGCN_IJCAI-18}\label{pemsd7ml}},PeMSD7L\textsuperscript{\ref{pemsd7ml}}, PEMSD10\cite{T-69}, PEMS-50/142/228\cite{T-109} 是研究交通速度预测使用的,将不会被本文统计在内。值得注意的是,研究者通常也对 PeMSDx($x=3,4,7,8$) 这 $4$ 个数据集会称呼成 PeMS0x,即例如 PeMSD4 又名为 PeMS04,在本文我们用前者来称呼。

% 是否提供图片,在内容整理.md里有图

下面将具体介绍 PeMS 的一些子集。使用广泛的四个子集如下所示。

\begin{description}
    \item[PEMSD3]\footnote{\url{https://github.com/guoshnBJTU/ASTGNN}\label{pems0x}} 使用了中北部区域(North Central Area)的 $358$ 个传感器,时间跨度为 $2018$ 年的 $9$ 月到 $11$ 月,共 $26208$ 个时间区间,每个区间 $5$ 分钟,刚好覆盖了 $91$ 天。数据缺失率为 $0.672\%$\cite{T-158}。
    \item[PEMSD4\textsuperscript{\ref{pems0x}}] 使用了旧金山湾区(San Francisco Bay Area)的 $307$ 个传感器,时间跨度为 $2018$ 年 $1$ 月到 $2$ 月,共 $16992$ 个时间区间,每个区间 $5$ 分钟,刚好覆盖了 $59$ 天。数据缺失率为 $3.182\%$\cite{T-158}。
    \item[PEMSD7\textsuperscript{\ref{pems0x}}] 使用了洛杉矶区域(Los Angeles Area)的 $883$ 个传感器,时间跨度为 $2017$ 年的 $5$ 月到 $8$ 月,共 $28224$ 个时间区间,每个区间 $5$ 分钟,覆盖了 $98$ 天。数据缺失率为 $0.452\%$\cite{T-158}。
    \item[PEMSD8\textsuperscript{\ref{pems0x}}] 使用了圣贝纳迪诺区(San Bernardino Area)的 $170$ 个传感器,时间跨度为 $2016$ 年的 $7$ 月到 $8$ 月,共 $17856$ 个时间区间,每个区间 $5$ 分钟,刚好覆盖了 $62$ 天。数据缺失率为 $0.696\%$\cite{T-158}。
\end{description}

绝大部分研究者使用 PeMS 数据集时,都会采用上述 $4$ 的数据集的全部或部分,然而,仍有研究者会利用 PeMS 的其他子集。下面展示 PeMS 的一部分其他子集。

\begin{description}
    \item[PEMS-S\cite{T-140}] 收集了 $2018$ 年 $5$ 月到 $6$ 月的所有工作日数据,每个时间区间 $5$ 分钟,$228$ 个传感器节点。%未知是不是flow
    \item[PEMS3-Stream\cite{T-310}] 收集了中北部区域 $2011$ 年到 $2017$ 年的 $7$ 月 $10$ 日到 $8$ 月 $9$ 日的数据,可用以研究同一地区长期的交通流量变化。
    \item[BayArea\cite{T-151}] 不同于著名的 PEMS-BAY,BayArea 数据集提供海湾地区(Bay Area)在 $2019/1/1\sim 2019/12/30$ 的 $699$ 个节点的流量数据。
    \item[PeMS\cite{T-174}] 该子集没有命名,提供 $5$ 分钟时间区间的 7 区(洛杉矶)的 $2015/1/1\sim 2015/5/31$ 的 $1681$ 个探测器数据。
    \item[SR99-S、I980-E\cite{T-181}] 从 10 区的 SR99-S 高速公路与 4 区奥克兰市(Oakland City)的 I980-E 街道在 $2017/9/4\sim2017/9/8$ 的数据。
    \item[I-405\cite{T-222}] 欧文中心大道(Irvine Center Drive)I-405 主干道第 2 车道的 $2014$ 年 $8$ 月的数据。
    \item[PEMS97、PEMS140\cite{T-348}] 该子集可以用于研究 COVID-19 疫情对交通流量的影响。提供了 $97+140$ 个节点在 $2020/4/1\sim2020/5/25$ 的 $30$ 天 $8640+7200$ 个长为 $5$ 分钟的时间段数据,并划分为了两个日期区间不同的子集 PEMS97、PEMS140 用于训练和测试。
    \item[LargeST]\footnote{\url{https://github.com/liuxu77/LargeST}} 整合自 PeMS,由 $4$ 个子集构成,分别位于美国加州(CA)、大洛杉矶(GLA)、大湾区(GBA)和圣地亚哥(SD),提供 $2017\sim2021$ 的 $5$ 分钟采样间距数据集,提供流量、速度等数据\cite{T-371}。
\end{description}

% \subsubsection{METR-LA}
% \subsubsection{Seattle Loop}
此外,多数研究者会将 METR-LA\textsuperscript{\ref{pems-bay}} 与 PeMS-BAY 数据集一并使用,以研究交通速度预测问题。且 Seattle Loop\footnote{\url{https://github.com/zhiyongc/Seattle-Loop-Data}} 也是一个常见的交通速度数据集。然而,本文只研究交通流量预测,对这两个数据集的介绍可以参考这篇文献\cite{T-ZS2}。

\subsubsection{出租车数据}\label{subsubsection:taxidata}

出租车数据是通过公司或政府通过收集出租车运行时的车辆位置轨迹数据(通常是 GPS 数据),并经过研究者处理得到的时间序列数据。这类数据的特点是轨迹分布广泛,需要跟具体空间信息结合。由于轨迹不局限于固定的道路,难以穷举,研究者往往使用网格空间节点建模。当一条车辆轨迹在某个时间段经过某个网格区域时,该车辆给该时间段该区域贡献流量。

轨迹数据与道路探测器等定点收集的数据相比,具有覆盖面广、对网格空间更适配、结果易于可视化和解释的优点,但相较下存在预处理繁琐、难以适用拓扑空间、数据质量难以保证、时间跨度短的不足\cite{T-ZS1}。

此外,出租车数据经常将流量细分为流入流量和流出流量(inflow and outflow),即对于每个网格节点的每个时间段,分别使用 $2$ 个数值记录流入流量与流出流量。但在拓扑图节点数据集上,通常不会区分流入与流出流量\cite{T-ZS45}。

在所有出租车数据中,有两个数据集使用较为广泛

\begin{description}
    \item[TaxiNYC]\footnote{\url{https://www.nyc.gov/html/tlc/html/about/trip_record_data.shtml}} 提供了美国纽约市(New York)从 $2009$ 年开始一直更新到现在的黄色、绿色出租车等数据,包含起止时间位置、距离等\cite{T-ZS2}。该数据集可提供流入与流出数据,通常预处理成网格形式。同 PeMS 一样,该数据集有广泛的子集,然而研究者们并未集中主要使用某些子集,往往每个子集只有一到数个研究者使用,难以统一分析。这里只详细介绍两个代表性的子集:\textbf{NYCTaxi14\footnote{\url{https://github.com/liulingbo918/ATFM/tree/master/data/TaxiNYC}}} $15\times5=75$ 个网格节点,长为 $30$ 分钟的时间区间,时间跨度
    为 $2014/1/1\sim2014/12/31$\cite{T-ZS45},该子集也可以用于做交通需求分析\cite{T-153}。\textbf{NYCTaxi15} $10\times20=200$ 个网格节点,长为 $30$ 分钟的时间区间,时间跨度为 $2015/1/1\sim2015/5/1$\cite{T-ZS45},而部分研究者\cite{T-162, T-124}使用了 $2015/1/1$ 到 $2015/3/1$ 的数据。除了上述子集外,还有研究者使用了其他的不同子集。例如,Sun 等人\cite{T-203}使用了 $2011/1/1\sim2016/6/30$ 的数据,Zhang 等人\cite{T-204}使用了 $2011\sim2014$ 年的数据,Fang 等人\cite{T-274}使用了 $2015$ 全年的数据,Bai 等人\cite{T-218}使用了 $2018/1\sim2018/4$ 的数据,Zhang 等人\cite{T-291}使用了 $2018\sim2019$ 两年的数据,Fang 等人\cite{T-301}使用了 $2016/1\sim2016/6$ 的数据,Yao 等人\cite{T-328} 使用了 $2015/1/1\sim2015/7/1$ 的数据和其他多地数据(DC、Porto、CHI、BOS),Deng 等人\cite{T-354} 只使用了黄色出租车数据,它们都用于流量预测。由于子集庞多但使用不广,这里并未列表统计这些子集,对这些年份数据研究感兴趣的读者可以查阅对应文献。
    %除了上述子集外,还有 $2015$ 年 $1$ 月到 $3$ 月的 $NYCTaxi15$\cite{T-162, T-124}、
    % \begin{itemize}
        % \item 
        % \item \textbf{NYCTaxi16} $16\times12=192$ 个网格节点,长为 $30$ 分钟的时间区间,时间跨度为 $2016/1/1\sim2016/2/29$\cite{T-ZS45}。
    % \end{itemize}
    % 然而,更多研究者使用 TaxiNYC 的数据集的其他子集,并形成了多样的不同子集,将它们详尽收录将较为冗长,因此这里仅作一些典例列举。
    % TODO check 数据格式
    \item[DIDI GAIA]\footnote{\url{https://outreach.didichuxing.com/}} 滴滴盖亚开发数据集,统计了滴滴平台在多个城市的网约车行驶传感器数据,包括车辆平均速度、起点终点、轨迹、驾驶场景等\cite{T-ZS2}。该数据集的子集众多,研究者们没有很集中地使用某些子集。尽管该数据集更多用于交通需求预测,但也可用于交通流量预测。下面按城市对本文收集的子集进行列举。
    \begin{description}
        \item[成都] Zheng 等人\cite{T-177}使用了成都 $2016/11$ 的 $69049$ 个 POI 的数据;Liu 等人\cite{T-297}使用了 $2016/12$ 的 $32\times32=1024$ 网格节点的 $14400$ 个时间段数据,每个时间段 $3$ 分钟(\textbf{Didi-Chengdu})。Stromann 等人\cite{T-265}、Yuan 等人\cite{T-293}也用了成都数据。
        \item[西安] Wang 等人\cite{T-343}使用西安每 $3$ 秒收集一次的 $2016/10$ 的共 $3784063$ 条轨迹数据,划分为 $550m\times550m$ 大小的网格区域汇总为 $5$ 分钟的流量数据(\textbf{TaxiXi'an})。Yuan 等人\cite{T-293}也用了成都数据。
        \item[南京] Chen 等人\cite{T-270} 使用了南京数据
        \item[深圳] Lu 等人\cite{T-279} 使用了深圳数据  
        \item[北京] Shi 等人\cite{T-306} 使用了北京、上海的 $2018/11/8\sim2018/12/30$ 的数据分别划分 $17\times10$、$18\times9$ 网格,整合成 $1$ 小时粒度预测起点-终点流量(Origin-Destination Flow)\textbf{Didi-Beijing、Didi-Shanghai}
        \item[上海] Shi 等人\cite{T-306} 使用了北京、上海数据(见上一条)。
    \end{description}
    由于地点众多,且同一个地点可能存在多个研究者的不同子集,并且暂未存在某些子集成为代表性主要研究者使用的子集,因此我们未对所有搜集的子集进行列表格统计。此外,还有研究者\cite{T-316,T-318} 预测北京、苏州、沈阳的预计到达时间。
\end{description}

下面将介绍本文搜集的其他出租车数据。有部分数据集的原始出处不明,也可能是 Didi GAIA 的子集。

\begin{description}
    \item[TaxiBJ]\footnote{\url{https://github.com/Echo-Ji/ST-SSL_Dataset}} 以 $32\times32=1024$ 个网格节点的流入和流出流量提供了中国北京超过 $34000$ 辆出租车在下面四个时间段每 $30$ 分钟一次共 $22459$ 个有效时间段的 GPS 轨迹数据:(1)$2013/7/1\sim2013/10/30$、(2)$2014/3/1\sim2014/6/30$、(3)$2015/3/1\sim2015/6/30$、(4)$2015/11/1\sim2016/4/10$\cite{T-ZS2}。该数据集是广泛被使用的一个出租车数据集,最初由 Zhang 等人在 $2017$ 年提出\cite{T-51}。然而,许多研究者只使用该数据集的上述子集或北京出租车其他子集,如一部分日期段\cite{T-274}、时间段(如不使用深夜)\cite{T-205}。例如,Wang 等人\cite{T-343}使用 $800m\times600m$ 大小的网格的 $2015$ 上半年数据作为子集。
    \item[TaxiCD]\footnote{\url{https://js.dclab.run/v2/cmptDetail.html?id=175}} 提供了中国成都在 $2014/8$ 的 $14864$ 辆出租车 $14$ 亿条 GPS 数据,每条数据由出租车 ID、经纬度、时间戳和是否空车\cite{T-ZS2}。有研究者\cite{T-366}也使用了该数据进行车辆到达时间预测。%,但未转化为时间序列数据。
    % 该数据为原始数据,需要进一步处理。也有研究者提供了处理 \cite{O-32}。 TODO 可以 check 是否未转化
    % 共有 $3636845$ 次出行,时间为 $2014/8/3\sim2014/8/30$\cite{T-307}。
    \item[TaxiJN\cite{T-314}] 中国济南出租车交通流数据,有 $32\times16=512$ 个网格节点,共 $3323$ 个样本数,时间范围为 $2017/9/1\sim2018/1/31$。
    \item[TaxiGY\cite{T-314}] 中国贵阳出租车交通流数据,有 $20\times24=480$ 个网格节点,共 $5270$ 个样本数,时间范围为 $2018/10/1\sim2019/5/26$。
    % \item[TaxiXI'AN\cite{T-343}] 中国西安 GAIA 数据集的每 $3$ 秒收集一次的 $2016/10$ 的共 $3784063$ 条轨迹数据,划分为 $550m\times550m$ 大小的网格区域汇总为 $5$ 分钟的流量数据。
    \item[T-drive]\footnote{\url{https://www.microsoft.com/en-us/research/publication/t-drive-trajectory-data-sample/}} 提供了中国北京市的 $3.3$ 万余辆出租车在 $2015/2/1\sim2015/6/30$ 的轨迹数据,时间段区间是 $60$ 分钟,由微软亚洲研究院、北京大学发布(Microsoft Asia Research Institute and Peking University),并提供 $32\times32$ 个网格节点的流入和流出流量\cite{T-151}。在 LibCity 开源代码库\cite{O-32} 中提供了对该数据集的预处理。
    % T-ZS2 说 6/2
    \item[SIP\cite{T-92}] 中国苏州工业园区(Suzhou Industrial Park,SIP)的交通流量、速度数据,还可以与新浪微博交通事故数据集结合进一步分析,有 $108$ 个区域和 $1399k$ 条原始数据\cite{T-92}。可以形成 $12\times5=60$ 网格节点,$30$ 分钟时间段间隔的 $2017$ 年 $1$ 月到 $3$ 月数据,也可以分析交通需求\cite{T-283}。
    \item[TaxiCHI]\footnote{\url{https://data.cityofchicago.org/Transportation/Taxi-Trips-2016/bk5j-9eu2}} 美国伊利诺伊州芝加哥市(Chicago, Illinois, USA)的 $2016/03/01\sim2016/04/30$ 的数据,使用 $11\times21=231$ 的网格节点\cite{T-301}。
    \item[TaxiPorto]\footnote{\url{https://www.kaggle.com/c/pkdd-15-predict-taxi-service-trajectory-i}} 葡萄牙波尔图(Porto, Portugal) 每 $15$ 秒收集一次并汇总成 $60$ 分钟的 $2013/7/1\sim2014/6/3$ 的流量数据,网格区域大小是 $500m\times500m$,共 $1891921$ 条轨迹数据\cite{T-343}。
    \item[SG-TAXI\cite{T-292}] 新加坡超过 $2$ 万出租车的 GPS 轨迹数据,由新加坡陆路交通管理局(Singapore Land Transport Authority)提供,收集了 $8$ 周 $2016/3/14\sim2016/5/8$ 的数据,由 $2404$ 个路段组成。与处理后组成 $15$ 分钟时间段的交通流数据。
    \item[Taxi-Rome]\footnote{\url{https://crawdad.org/roma/taxi/20140717}} 超过 $30$ 天的 $320$ 辆出租车在意大利罗马的 GPS 轨迹数据,$32\times32=1024$ 个网格节点,$3$ 分钟的时间段跨度,$14400$ 条数据\cite{T-297}。
\end{description}

此外,对轨迹数据,还有研究单车流量与需求的数据如 BikeNYC\footnote{\url{https://www.citibikenyc.com/system-data}}、BikeDC\footnote{\url{https://www.capitalbikeshare.com/system-data}}、BikeChicago\footnote{\url{https://www.divvybikes.com/system-data}} 等,以及研究地铁人流量的 HZMetro\footnote{\url{https://github.com/ivechan/PVCGN}\label{pvcgn}}、SHMetro\textsuperscript{\ref{pvcgn}} 等,以及一些交通速度预测的知名数据集:SHSpeed\footnote{\url{https://github.com/xxArbiter/grnn}}、SZ-Taxi\footnote{\url{https://github.com/lehaifeng/T-GCN}}。由于本文着重研究车流量,不对它们进行详细介绍。

\subsubsection{北京数据}
在剩下未介绍的交通车流量数据集里,中国北京的数据集占了许多,所以专门开一节列表介绍所有本文收集的北京数据集。然而,并不清楚是否这些数据集来源于同一个数据来源\cite{T-ZS1}。

\begin{description}
    \item[TaxiBJ] 参见\ref{subsubsection:taxidata} 一节。
    \item[Didi-Beijing] 参见\ref{subsubsection:taxidata} 一节。
    \item[Daxing\cite{T-183}] 北京市大兴区(Daxing District) $88$ 个道路探测器每 $5$ 分钟收集一次的 $2019/9$ 的流量数据。
    \item[TrafficBJ\cite{T-123}] 北京五环路,从百度地图获取的交通拥挤状况数据,能反映交通流量。收集了 $2017/5/14\sim2017/12/5$ 的每 $6$ 分钟一次,$6:00\sim23:00$ 的数据,共 $36040$ 个时间区间,将 $904$ 条道路连接转为了 $53\times42$ 个网格点和连通矩阵。矩阵值在 $1\sim 4$ 之间代表拥挤程度不断增加。
    \item[TaxiBeijing\cite{T-188}] 未命名数据集。北京五环,预处理成 $81$ 个拓扑图节点,时间区间 $10$ 分钟,共 $27$ 天的数据。初始是 $555m\times425m$ 的 $8x\times80$ 区域的 $2009/5/1\sim2009/5/29$ 的数据。
    \item[GT-221、WRS-393、ZGC-564\cite{T-256}] 由北京市交通运输委员会(Beijing Municipal Commission of Transport)收集的 $2015/4/1\sim2015/7/31$ 的 $4$ 万出租车的每 $5$ 分钟的数据集。选用其中的三个子集:(1)GT-221:工体(Workers’ Stadium)的 $221$ 个路段;(2)WRS-393:北京西站(Beijing West Railway Station)的 $393$ 个路段;(3)ZGC-564:中关村(China Silicon Valley)的 $564$ 个路段。
    \item[W3-715、E5-2907\cite{T-319}] 西三环(West 3rd Ring Road)的 $715$ 个路段与东五环(East 5th Ring Road)的 $2907$ 个路段,$2018/12/24\sim2019/4/21$ 每天 $6:00\sim22:00$ 的工作日数据,共 $10$ 周。从 Amap 收集的匿名数据。
    \item[Beijing\cite{T-325}] 提供了北京的三种数据:(1)地铁:$278$ 节点,$10$ 分钟时间段长度,$2016/6/1\sim2016/6/29$;(2)公交:$24$ 节点,$60$ 分钟时间段长度,$2016/6/1\sim2016/6/29$;(3)出租车:$198$ 节点,$20$ 分钟时间段长度,$2015/11/1\sim2016/5/31$。都是每天 $6:00\sim22:00$ 的流量数据。
\end{description}

北京的非车流量数据集包括但不限于交通速度的 Q-Traffic\footnote{\url{https://github.com/JingqingZ/BaiduTraffic}}、NE-BJ\cite{T-134}、Beijing Traffic\footnote{\url{https://github.com/deepkashiwa20/Urban_Concept_Drift}}、Urban-BJ\cite{T-202}、Ring-BJ\cite{T-202} 和其他数据集\cite{T-190,T-214,T-247, T-140},地铁数据\cite{T-209},自行车数据 mobileBJ\cite{T-311},交通需求数据集\cite{T-234, T-327, T-359},时间预测数据集\cite{T-235, T-362},这里不详细介绍。

\subsubsection{其他数据集}

除了上述数据集外,本文收集的其余数据集列表如下。

\begin{description}
    \item[UVDS\cite{T-370}] UVDS(Urban Vehicle Detection System Dataset) 由 $104$ 个探测器拓扑图节点在韩国大田广域市(Daejeon, Korea)\cite{T-369} 等整合而来的。其中大田有 $21$ 个定点探测器,每 $5$ 分钟收集一次共 $2017/4/1\sim2017/4/30$ 的数据,提供了交通流量、速度、占用率等数据。
    \item[Highways England]\footnote{\url{http://tris.highwaysengland.co.uk/detail/trafficflowdata}} 英国政府提供的英国M、A级高速公路数据集,采样间隔为 $15$ 分钟,有流量、速度、时间、占用率等交通特征\cite{T-ZS35}。Shi 和 Du\cite{T-186}使用了 M25 j20-j21A, M1 j6-j6A, A405, M25 j21A-j22 和 M1 j6A-j8 道路的 $2021/7$ 的数据。Han 等人\cite{T-300}使用了 $2014/1\sim2014/6$ 的 $314$ 个节点的数据。
    % \item[Urban]\footnote{\url{http://topis.seoul.go.kr/}} 记录了韩国首尔江南区(Urban1)和麻浦区(Urban2)的流量数据, 基于 GPS 采集 $7$ 万多辆车辆轨迹数据,时间间隔\cite{T-73}。
    \item[Xiamen] Zheng 等人\cite{T-143} 使用 $95$ 个探测器在 $2015/8/1\sim2015/12/31$ 获取的 $5$ 分钟采样时间段数据,他们的团队在另一个研究\cite{T-177} 使用了另外的时间段数据。
    \item[Traffic Flow Prediction\cite{T-152}] 美国北弗吉尼亚州/华盛顿特区(Northern Virginia/Washington, D.C.)的两条主要高速公路的 $15$ 分钟采样间距的 $36$ 个传感器的 $2017/1/2\sim2017/2/1$ 的数据。
    \item[NYC\cite{T-196}] 美国纽约在 $2018/10/1\sim2018/12/31$ 提供 $1km\times1km$ 单位大小 $32\times32$ 网格,$36$ 条地铁线,$10380$ 个 POI 的 $13248$ 个时间段的交通流矩阵,时间间隔是 $10$ 分钟。
    % 市交通部(\textbf{N}ew \textbf{Y}ork \textbf{C}ity \textbf{D}epartment \textbf{o}f \textbf{T}ransportation)
    \item[Chicago\cite{T-196}] 在 $2018/10/1\sim2018/12/31$ 提供 $1km\times1km$ 单位大小 $32\times32$ 网格,$8$ 条地铁线,$9625$ 个 POI 的 $13248$ 个时间段的交通流矩阵,时间间隔是 $10$ 分钟。
    \item[SCATS\cite{T-44}] 澳大利亚墨尔本维多利亚街(Victoria Street, Melbourne)的 $2016$ 年非节假日的 $5$ 分钟数据,经过标准化的流量数据,可以研究早高峰,$30$ 个路段 $24$ 个交叉口。
    \item[Traffic4cast]\footnote{\url{https://www.iarai.ac.at/traffic4cast/}} 俄国莫斯、科柏林、伊斯坦布尔(Berlin, Moscow, and Istanbul)的 $2019$ 年的 GPS 轨迹数据,划分为 $100m\times100m$ 的 $5$ 分钟采样间隙的交通流量、数据等信息\cite{T-206}。
    \item[Henan\cite{T-223}] 未命名数据集。河南省公路管理系统(Henan Highway Management System)在高速收费站收集的 $2017/3\sim2017/9$ 数据,超过 $2$ 万条数据,$37$ 条高速,$279$ 个收费站,时间区间可以取 $5$ 分钟等\cite{T-372}。
\end{description}

此外的其他类型交通数据集,比如人流量预测、速度预测等,它们并不在本文研究范围内,故不作详细讨论,下面仅枚举一部分如下,以供有需要的读者使用。
\begin{itemize}
    \item 人流量。美国纽约地铁\cite{T-290}、HappyValley\cite{T-205}、中国重庆\cite{T-220}、成都\cite{T-274}、室内人流量\cite{T-272}。
    \item 非机动车流量:共享电动踏板车(DES, dockless e-scooter sharing)\cite{T-286}
    \item 速度数据集:韩国首尔\cite{T-73, T-236},日本东京EXPY-TKY\cite{T-136}、美国洛杉矶 Los-Loop\cite{T-218}、纽约\cite{T-109, T-92}、西雅图 INRIS-SEA\cite{T-79}、中国杭州 HZJTD\cite{T-69, T-334, T-244}、天津\cite{T-190}、香港\cite{T-221}。
    \item 时间预测数据集:中国北京\cite{T-322}、成都\cite{T-322}、西安\cite{T-322}。
    \item 事故风险数据集:韩国首尔\cite{T-332}、美国纽约\cite{T-289}、美国芝加哥\cite{T-289}。
    \item 需求数据集:美国芝加哥等\cite{T-278, T-255}、中国甘肃兰州\cite{T-226}、北京\cite{T-234}、上海\cite{T-234}、EVCARD\cite{T-70}、深圳\cite{T-268}、西安\cite{T-268}、成都等\cite{T-268}。
    \item 停车预测数据集:澳大利亚墨尔本\cite{T-273}。
\end{itemize}

\subsection{总结}
本文收集的时间序列序列数据集如表 \ref{table:timedataset} 所示。其中粒度是指每个时间段的长度,单位为分钟,例如时间粒度为 $5$ 代表每五分钟统计一次流量数据。数据集名称上标'的,代表该数据集为一个未命名子集,使用了原始数据的命名,或者多个子集重名。(TODO 举例分析解读表格)

\setlength{\tabcolsep}{2pt}
% \setlength\LTleft{-1in}
% \setlength\LTright{-1in plus 1 fill}
\begin{footnotesize}
\begin{longtabu} to \paperwidth {|c|c|c|c|c|c|c|} % {|X|X|X|X|X|X|X|}
    \caption{交通流量预测的时间序列数据集表格}\label{table:timedataset}
    \\ \hline
    出处 & 数据集名称 & 时间范围 & 地理位置 & 时间粒度 & 时间段数 & 节点数量 \\ \hline
    \endfirsthead
 
    \caption{交通流量预测的时间序列数据集表格(续表)}
    \\ \hline
    出处 & 数据集名称 & 时间范围 & 地理位置 & 时间粒度 & 时间段数 & 节点数量 \\ \hline
    \endhead

    \hline
    \multicolumn{7}{|r|}{\textit{接下页}} \\ \hline % Continued on next page
    \endfoot
 
    \hline
    \endlastfoot

    \cite{T-81} & PEMSD3 & $2018/9/1\sim2018/11/30$ & 美国加州 & $5$ 分钟 & $26208$ & $358$ \\ \hline
    \cite{T-81} & PEMSD4 & $2018/1/1\sim2018/2/28$ & 美国加州 & $5$ 分钟 & $16992$ & $307$ \\ \hline
    \cite{T-81} & PEMSD7 & $2017/5/1\sim2017/8/31$ & 美国加州 & $5$ 分钟 & $28224$ & $883$ \\ \hline
    \cite{T-81} & PEMSD8 & $2016/7/1\sim2016/8/31$ & 美国加州 & $5$ 分钟 & $17856$ & $170$ \\ \hline
    \cite{T-140} & PEMS-S & $2018/5/1\sim2018/6/30$ 的工作日 & 美国加州 & $5$ 分钟 & $17568$ & $228$ \\ \hline 
    % 17568=24*12*61推断 TODO
    \cite{T-310} & PEMS3-Stream & $2011\sim2017$ 的 $7/10\sim8/9$ & 美国加州 &  & $57600$ & $448$ \\ \hline
    \cite{T-151} & BayArea & $2019/1/1\sim2019/12/30$ & 美国加州 &  &  & $699$ \\ \hline
    \cite{T-174} & PeMS' & $2015/1/1\sim2015/5/31$ & 美国加州 & $5$ 分钟 & $8928$ & $1681$ \\ \hline
    % 8928=12*24*31推断 TODO
    \cite{T-181} & SR99-S、I980-E & $2017/9/4\sim2017/9/8$ & 美国加州 &  &  &  \\ \hline
    \cite{T-222} & I-405 & $2014/8/1\sim2014/8/31$ & 美国加州 &  &  &  \\ \hline
    \cite{T-348} & PEMS97 & $2020/4/1\sim2020/5/1$ & 美国加州 & $5$ 分钟 & $8640$ & $97$ \\ \hline
    \cite{T-348} & PEMS140 & $2020/5/1\sim2020/5/25$ & 美国加州 & $5$ 分钟 & $7200$ & $140$ \\ \hline
    \cite{T-51} & TaxiBJ & $2013/7/1\sim2016/4/10$ 的四个子集 & 中国北京 & $30$ 分钟 & $22459$ & $1024$ \\ \hline
    \cite{T-307} & TaxiCD & $2014/8/3\sim2014/8/30$ & 中国成都 &  &  &  \\ \hline
    \cite{T-367} & TaxiNYC14 & $2014/1/1\sim2014/12/31$ & 美国纽约 & $30$ 分钟 & $17520$ & $75$ \\ \hline
    % 17520 是 h5py 读 https://github.com/liulingbo918/ATFM/blob/master/data/TaxiNYC/NYC2014.h5 行数
    \cite{T-153} & TaxiNYC15 & $2015/1/1\sim2015/5/1$ & 美国纽约 & $30$ 分钟 & $5760$ & $200$ \\ \hline
    % 5760=24*2*120,也是 https://github.com/tangxianfeng/STDN 里统计行数sum(2,x)得到的
    \cite{T-297} & Didi-Chengdu & $2016/12/1\sim2016/12/31$ & 中国成都 & $3$ 分钟 & $14400$ & $1024$ \\ \hline
    \cite{T-279} & Didi-Beijing & $2018/11/8\sim2018/12/30$ & 中国北京 & $60$ 分钟 &  & $170$ \\ \hline
    \cite{T-279} & Didi-Shanghai & $2018/11/8\sim2018/12/30$ & 中国上海 & $60$ 分钟 &  & $161$ \\ \hline
    \cite{T-97} & T-drive & $2015/2/1\sim2015/6/30$ & 中国北京 & $60$ 分钟 & $6300$ & $1024$ \\ \hline
    \cite{T-301} & TaxiCHI & $2016/03/01\sim2016/04/30$ & 美国芝加哥 &  &  & $231$ \\ \hline
    \cite{T-92} & SIP & $2017/1/1\sim2017/3/31$ & 中国苏州 & $30$ 分钟 & $4320$ & $60$ \\ \hline
    %4320=48*90
    \cite{T-292} & SG-TAXI & $2016/3/14\sim2016/5/8$ & 新加坡 & $15$ 分钟 & $6240$ & $2404$ \\ \hline
    %6240=24*4*65
    \cite{T-297} & Taxi-Rome & $2014/7/17$ 超过 $30$ 天 & 意大利罗马 & $3$ 分钟 & $14400$ & $1024$ \\ \hline
    \cite{T-297} & Didi-Chengdu & $2016/12/1\sim2016/12/31$ & 中国成都 & $3$ 分钟 & $14400$ & $1024$ \\ \hline
    \cite{T-314} & TaxiJN & $2017/9/1\sim2018/1/31$ & 中国济南 &  & $3323$ & $512$ \\ \hline
    \cite{T-314} & TaxiGY & $2018/10/1\sim2019/5/26$ & 中国贵阳 &  & $5270$ & $480$ \\ \hline
    \cite{T-343} & TaxiXi'an & $2016/10/1\sim2016/10/31$ & 中国西安 & $5$ 分钟 & &  \\ \hline
    \cite{T-343} & TaxiProto & $2013/7/1\sim2014/6/3$ & 葡萄牙波尔图 & $60$ 分钟 & &  \\ \hline
    \cite{T-183} & Daxing & $2019/9/1\sim2019/9/30$ & 中国北京 & $5$ 分钟 & $8640$ & $88$  \\ \hline
    % 8640=30*24*5
    \cite{T-123} & TrafficBJ & $2017/5/14\sim2017/12/5$ & 中国北京 & $6$ 分钟 & $36040$ & $1806$  \\ \hline
    \cite{T-188} & Beijing' & $2009/5/1\sim2009/5/29$ & 中国北京 & $10$ 分钟 & $4176$ & $81$  \\ \hline
    % 4176=24*10*29
    \cite{T-256} & GT-221 & $2015/4/1\sim2015/7/31$ & 中国北京 & $5$ 分钟 & $34848$ & $1221$  \\ \hline
    \cite{T-256} & WRS-393 & $2015/4/1\sim2015/7/31$ & 中国北京 & $5$ 分钟 & $34848$ & $221$  \\ \hline
    \cite{T-256} & ZGC-564 & $2015/4/1\sim2015/7/31$ & 中国北京 & $5$ 分钟 & $34848$ & $564$  \\ \hline
    % 34848=121*24*12
    \cite{T-319} & W3-715 & $2018/12/24\sim2019/4/21 6:00\sim24:00$ 工作日 & 中国北京 & &  & $2907$  \\ \hline
    \cite{T-319} & E5-2907 & $2018/12/24\sim2019/4/21 6:00\sim24:00$ 工作日 & 中国北京 & &  & $2907$  \\ \hline
    \cite{T-325} & Beijing' & $2016/6/1\sim2016/6/29 6:00\sim22:00$ & 中国北京 & $10$ 分钟 & $3132$ & $278$  \\ \hline
    % 3132 = 18*6*29
    \cite{T-325} & Beijing' & $2016/6/1\sim2016/6/29 6:00\sim22:00$ & 中国北京 & $10$ 分钟 & $522$ & $24$  \\ \hline
    % 522 = 18*1*29
    \cite{T-325} & Beijing' & $2015/11/1\sim2016/5/31 6:00\sim22:00$ & 中国北京 & $10$ 分钟 & $3132$ & $198$  \\ \hline
    % 11448 = 18*3*212
    \cite{T-369} & UVDS & $2017/4/1\sim2017/4/30$ & 韩国大田 & $5$ 分钟 & $8640$ & $21$ \\ \hline
    % 8640=30*24*12
    \cite{T-186} & Highways England' & $2017/1/1\sim2017/7/31$ & 英国 & $15$ 分钟 & $2976$ & $3$ \\ \hline
    % 2976=4*24*31
    \cite{T-300} & Highways England' & $2014/1/1\sim2014/6/30$ & 英国 & $15$ 分钟 & $17280$ & $314$ \\ \hline
    % 17280=30*24*12
    \cite{T-143} & Xiamen & $2015/8/1\sim2015/12/31$ & 中国厦门 & $5$ 分钟 & $44064$ & $95$ \\ \hline
    % 44064=153*12*24
    \cite{T-143} & TrafficFlowPrediction & $2017/1/2\sim2017/2/1$ & 美国 & $15$ 分钟 & $2880$ & $36$ \\ \hline
    % 30*4*24
    \cite{T-196} & NYC' & $2018/10/1\sim2018/12/31$ & 美国纽约 & $10$ 分钟 & $13248$ & $1024$ \\ \hline
    \cite{T-196} & Chicago' & $2018/10/1\sim2018/12/31$ & 美国芝加哥 & $10$ 分钟 & $13248$ & $1024$ \\ \hline
    \cite{T-44} & SCATS & $2016/1/1\sim2016/12/31$ & 澳大利亚 & $5$ 分钟 & $73152$ & $30$ \\ \hline
    % 254*24*12, 254是poe说2016澳洲工作日天数
    \cite{T-206} & Traffic4cast & $2019/1/1\sim2019/12/31$ & 俄罗斯 & $5$ 分钟 & & \\ \hline
    % 105120=24*12*365
    \cite{T-223} & Henan' & $2017/3/1\sim2017/9/30$ & 中国河南 & $5$ 分钟 & $20000+$ & $279$ \\ \hline
\end{longtabu}
\end{footnotesize}



\section{其他辅助数据介绍}
介绍

\subsection{日期与时间}
介绍

\subsection{天气}
介绍

\subsection{交通异常事件}
介绍

\subsection{其他数据}
介绍

\bibliographystyle{unsrt}
\bibliography{../article/thesis.bib}
\end{document}