\documentclass{ctexart}
\usepackage{hyperref}
\begin{document}
\title{交通流量预测数据集综述}
\author{中山大学}
\date{\today}
\maketitle
\tableofcontents
\section{概述}

交通流量预测是一种时空序列预测任务,其数据集主要由时间序列数据、空间信息数据和其他辅助数据三部分组成。

\begin{description}
    \item[空间信息数据] 主要为真实世界的道路结构进行建模提供必要信息。根据建模方式不同,可以将其分为网格空间信息和拓扑空间信息。无论是哪一种信息,都以节点作为基本单位。其中,网格空间信息将道路结构划分为若干个网格节点,每个节点表示一小个矩形区域,多个节点排列成二维矩阵来表示现实世界的一片区域。而拓扑空间信息则将道路视作有向图或无向图,图中的每个图节点可以表示一条道路、一个探测器、一个路口等。下图 展示了两种建模的空间信息示意图。
    \item[时间序列数据] 是交通流量预测的核心数据,由各节点的历史交通流量数据组成,每个节点通常按等间隔地记录过去的一段时间每单位时间段经过该节点的车流量大小数值。一般而言,每个节点的时间跨度、间隔完全一致,故时间段数相同。因此,通常将其表示为一个数值矩阵,行数为节点数量,列数为时间段的数目。在预测任务中,通常利用所有节点的多段过去的流量数据,来预测全体节点未来一段或多段的交通流量。
    \item[其他辅助数据] 这主要包括日期时间、天气、交通异常事件等辅助信息。这是由于节假日与工作日(不同日期)、早晚高峰与非高峰时段(不同时间)、晴天与暴雨天(不同天气)、交通事故等异常事件都会对交通流量产生影响。因此,将这些辅助信息作为补充信息提供给模型,可以提升预测效果。
\end{description}

这些数据集组成成分会存在相互影响。例如,时间序列数据与空间信息数据存在密不可分的关联,例如,两条道路相连通时,一条道路的堵塞,会在一段时间内造成另一条道路也发生堵塞。因此,当预测某个节点的未来交通流量时,不能只考虑该节点的时间序列数据,还要联系空间信息找到相关联的其他节点,将这些关联节点的时间序列数据也共同纳入考虑。

在本文的剩余部分,将详细介绍这三种组成成分,分析它们的特征。随后对每种成分,具体介绍国内外的常用数据集,陈述这些数据集的特点,并列表格详细展示本文所搜集的数据集及其相关统计。

\section{空间信息数据介绍}
空间信息数据有多种划分方式,例如,Tedjopurnomo 等人\cite{tedjopurnomo_survey_2022} 提出,按照数据收集的方式,可以将数据分为点数据集和轨迹数据集。其中,点数据集是由探测器等固定不动的仪器从固定点收集的数据,而轨迹数据为使用 GPS 等定位技术在车辆上收集的数据。而 Jiang 和 Luo\cite{jiang_graph_2023} 提出,按照道路的类型和粒度可以把交通流量数据分为道路级数据、区域级数据、站点级数据。而在本文中,将交通流量数据里的空间信息数据按照空间结构划分为了欧氏空间的网格型数据和非欧空间的拓扑型数据。

原始的交通路网数据可以直接从在线地图服务、政府部门网站等获得,如 OpenStreetMap\footnote{\url{https://www.openstreetmap.org}}。通常会将原始数据进行预处理,然后与时间序列数据等其他组成成分整理并形成公开数据集,并发布在网络上,例如 GitHub\footnote{\url{https://github.com/}} 和 PaperWithCode\footnote{\url{https://paperswithcode.com/task/traffic-prediction}}。

% \subsection{网格空间信息数据}
对网格空间,可以表示为一个矩阵,每个元素代表一片区域,每个区域大小相等,通常是正方形区域(da)。形式化地说,可以表示成 $r$ 行 $c$ 列的网格 $G$,且一般 $r,c$ 较为接近,即会选取一片正方形区域做为网格空间。通常网格数据使用 CNN 来做流量预测应用。

% \subsection{拓扑空间信息数据}
对拓扑空间,一张图 $G$ 由点集 $V$ 和边集 $E$ 组成,边集可以存储成邻接表等形式,更多地通过邻接矩阵 $A$ 的方式存储,因为矩阵更适合模型运算。根据边是否为单向的,可以分为有向图和无向图。边通常具有权重信息,如布尔值 $1$ 代表连通、$0$ 代表不连通,或者使用连续数值表示两点间的距离、关联程度或其他信息。常常使用 GNN 来实现拓扑空间数据的流量预测。%多数数据集将图视为静态的,也有些数据集是随时间而动态变化的。?
%注意cite来引用这些例子

\section{时间序列数据集介绍}
介绍

\section{其他辅助数据介绍}
介绍

\subsection{日期与时间}
介绍

\subsection{天气}
介绍

\subsection{交通异常事件}
介绍

\subsection{其他数据}
介绍

\bibliographystyle{unsrt}
\bibliography{../article/thesis.bib}
\end{document}